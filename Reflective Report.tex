% Please do not change the document class
\documentclass{scrartcl}

% Please do not change these packages
\usepackage[hidelinks]{hyperref}
\usepackage[none]{hyphenat}
\usepackage{setspace}
\usepackage{graphicx}
\doublespace

% You may add additional packages here
\usepackage{amsmath}

% Please include a clear, concise, and descriptive title
\title{Reflective Essay}

% Please do not change the subtitle
\subtitle{COMP120 - Reflective Report}

% Please put your student number in the author field
\author{1503048}

\begin{document}

\maketitle

\abstract{In this reflective report, I'll be reviewing my learning while reflecting on my projects and approaches for the first semester.}

\section{Introduction}

The three weakness I'll be exploring in this report are: time management, critical evaluation, . All of these skills are significant to me for professional practise and have been emphasised during my first semester.

\section{Time Management}

\subsection{Description - What happened? What is being examined?}

One of my weakness I'll be examining is time management. With all tasks it is crucial to follow a strict and thoughtful planning arrangement which accounts for all aspects of the project. For me, the relevance of this issue was for my final deadlines.

\subsection{How has this impacted my learning and projects?}

During my projects, I found that my allocated work time had been misplaced. I found most of the time was focused on research of ideas and techniques rather than practical applications. This not only lead to a low quality of work but also low quantity too. This even distribution also hindered the overall project.


\subsection{Interpretation - What is relevant about this idea?}

Initially, I noticed that I did not feel confident in my abilities and questioned what I could do to improve. For me, this meant that I enjoyed the course much less, which lead to frustration. At the time, I did not know how to overcome this issue which lead to an endless cycle of frustration and research.

\subsection{Outcome- What have I learned and what can be done in the future?}

Having discusses this issue with others, I now realise I should have notified my lecturers of my struggles. I have significantly changed my opinion of how programming works, and my knowledge of self teaching has improved. For the future, I aim to make strict plans on how to structure my time and for how long one task is expected to task. If the plan is not adhered to, then I must approach my lecturer or colleagues with my concerns.


\section{Critic Ability}

\subsection{Description - What happened? What is being examined?}

Another of my weakness is critical evaluation, which requires the objective review of both essays and coding work. It has been crucial to evaluate both my work and my peers. Also, the ability to critique your own work allows for issues to be discussed and improvements to be made.

\subsection{How has this impacted my learning and projects?}

While struggling to maintain confidence in my own work, I felt unprepared to evaluate others work. This also lead to myself being over critical of my own work, but in a way that wasn't constructive.

\subsection{Interpretation - What is relevant about this idea?}

For me, the significant issue arose from very low confidence. Subsequently, I felt I was not appropriately equipped to offer critical advice. In addition to this, I felt unfamiliar with marking others work with regards to coding practise.

\subsection{Outcome- What have I learned and what can be done in the future?}

Having applied this skill over several review sessions, I now realise that, while making reference to project objectives, evaluation is more structured and focused.

\section{Confidence}



%\bibliographystyle{ieeetr}
%\bibliography{PCG_export}

\end{document}